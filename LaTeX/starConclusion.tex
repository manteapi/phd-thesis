\section{Chapter conclusion}
In this chapter, we proposed an introduction to continuum mechanics and described a variety of methods for controlling simulations.
Most of the time, the practibility of these models and methods depends on the simulated phenomena and the required computational time.
Adaptivity is a general strategy to achieve large scale simulations and resolve complex behaviors.
In Chapter~\ref{chap:starAdaptivity}, we propose a detailed review of these models.
In Chapter~\ref{chap:arps}, we extend an adaptive model, initially proposed for nanosystems simulations, to speed-up particle simulations in computer graphics, 
Then, we study how to perform detailed topological changes with the frame-based model in Chapter~\ref{chap:cutting}, an interesting approach which ensures a very small number of degrees of freedom and therefore interactive performances.
Finally, we explore a new way of editing fluid animations in Chapter~\ref{chap:fluidsculpting}, where the user can manipulate sub-parts of the animation interactively without the need to re-simulate.
