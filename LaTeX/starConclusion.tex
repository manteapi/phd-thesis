\section{Chapter conclusion}
There is a large number of techniques and models that are used to improve the efficiency of simulations, to simulate new phenomena and to allow the user to control a simulation.
We begin our contributions by extending an adaptive model, initially proposed for nanosystems simulations, to speed-up particle simulations in computer graphics, in Chapter~\ref{chap:arps}.
Then, we study how to perform detailed topological changes with the frame-based model in Chapter~\ref{chap:cutting}, an interesting approach which ensures a very small number of degrees of freedom and therefore interactive performances.
Finally, we explore a new way of editing fluid animations in Chapter~\ref{chap:fluidsculpting}, where the user can manipulate sub-parts of the animation interactively without the need to re-simulate.
