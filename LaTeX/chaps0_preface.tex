%\chapter{Remerciements}
%\begin{itemize}
%    \item Collaboration with Wei-Lun Sun, PhD student at Berkeley. We worked together on how to perform detailed cutting in a frame-based simulation framework. Results were published at Motion In Games \cite{Manteaux2015}.
%    \item Collaboration with Julio Toss, PhD student at Inria. We shared knowledges about parallel computing and frame-based simulation.
%    \item Thibault Lejemble, Amélie Fondevilla, Thibault Blanc-Beyne, Nicolas Durin
%    \item Ulysse Vimont
%\end{itemize}

\chapter{R\'esum\'e}
En informatique graphique les phénomènes physiques simulés pour la création d'animation, de jeux vidéos ou le prototypage d'objet sont de plus en plus complexes. Tout d'abord en terme de coût calculatoire, l'échelle des simulations étant de plus en plus importante. Ensuite dans les phénomènes eux-mêmes qui requièrent des modèles de pouvoir changer d'état et de forme. Cette complexité grandissante introduit de nouveaux défis quand il s'agit d'offrir à un utilisateur un contrôle sur ces simulations à grande échelle. Dans de nombreux cas, ce contrôle est réduit à un processus d'essais et d'erreurs dédié à la détermination des paramètres de la simulation qui satisferont au mieux les objectifs de l'utilisateur.

Dans cette thèse, nous proposons trois techniques pour répondre en partie à ces défis. Tout d'abord nous introduisons un nouveau modèle adaptatif permettant de réduire le temps de calcul dans des simulations particulaires Lagrangienne. \{`}A l'inverse des méthodes de ré-échantillonnage, le nombre de degrés de liberté reste constant au cours de la simulation. La méthode est ainsi plus simple à intégrer dans un simulateur existant et la charge mémoire est constante ce qui peut être un avantage dans un contexte intéractif. Ensuite, nous proposons un algorithme permettant de réaliser la découpe détaillée d'objets fins et déformables. Notre méthode s'appuie sur une mise à jour dynamique des fonctions de forme associées à chaque degré de liberté, permettant ainsi de conserver un nombre de degré de liberté très faible tout en réalisant des changements topologiques détaillés. Enfin, nous nous intéressons au contrôle d'animation de fluide en s'inspirant de ce que l'on peut trouver en modélisation 3D. Dans ce système, l'utilsateur travaille directement avec le résultat d'une simulation, c'est à dire une suite de maillages représentant la surface du fluide. Des outils de sélection et d'édition spatio-temporelle adaptés des logiciels de sculptures lui sont proposés.

\chapter{Abstract}