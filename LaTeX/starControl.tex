\section{Control of physics-based animation}

\subsection{Classical animation vs. Physics-based animation}

\begin{itemize}
\item Designing an only physics-based animation is hard.
\item Boundary conditions need to be placed by hand.
\item Physical parameters need to be tweaked.
\item Reaching a specific pose at a specific time is a nightmare.
\item Each change of the initial configuration need a new run of the simulation to see the result.
\item It is tempting to create tools that allow a higher level of control on the design of a physics-based animation.
\end{itemize}

\begin{itemize}
\item Physics allow to resolve details that would be extremely hard to describe manually.
\item It is tempting to enhance an existing animation by adding physics.
\item It required to find a tradeoff between the artist-designed animation and a physical solution.
\end{itemize}

\begin{itemize}
\item Physics-based secondary motion
\item Example-based simulation
\item Guiding shapes for the global motion
\item Interactive trial/error editing of boundary conditions
\item Global deformation of existing animation
\end{itemize}

\subsection{General approaches}

\subsubsection{Mixing input animations with physics}

\paragraph{Control of rigid objects}
\cite{Popovic2000}, \cite{Popovic2003}

\paragraph{Control of deformable objects}
\cite{Kondo2005}, \cite{Bergou2007}, \cite{Martin2011}, \cite{Barbic2012}, \cite{Coros2012}, \cite{Hahn2012}, \cite{Hildebrandt2012}, \cite{Hahn2013}, \cite{Liu2013}, \cite{Li2014}

\subsubsection{Designing physics-based animation}

\subsubsection{Editing physics-based animation}
\cite{Pighin2004}

\subsection{Control of fluid simulation}

\subsubsection{Problems}

\subsubsection{Interactive design}

\subsubsection{Guiding shapes}

\subsubsection{Interactive editing}

The standard pipeline for controlling a simulation is mainly based on trial and error. Fluid parameters and boundary conditions are first set up by the user. The parameters are then tweaked and the simulation is re-run until reaching the expected behavior. As noticed previously, the control is indirect, for instance the user cannot control the trajectory nor the shape of the fluid. Moreover, the non-linear nature of the fluid behavior prevents the user from interactively controlling a low-resolution simulation and then achieving similar behaviors with the same parameters at a higher resolution.

To overcome these limitations, several methods propose to guide the fluid behavior by using geometric proxies which are easier to control than the high resolution fluid simulation variables themselves.
For example, artists can use a triangle mesh to specify a target shape for the fluid by adding artificial attraction forces based on the distance to the mesh surface. Similar approaches have been successfully developed to drive smoke~\cite{Fattal2004,Hong2004,Shi2005a} and liquid simulations~\cite{Shi2005b,Raveendran2012}. These meshes can also define specific keyframes from a global animation~\cite{Treuille2003,McNamara2004}.
Fluid trajectories can also be controlled with user-defined velocity fields~\cite{Kim2006}, distance fields~\cite{Yang2013}, or specific control particles~\cite{Thurey2006,Madill2013}. 

Taking this strategy further, the attracting surface itself can defined by a low-resolution fluid simulation. To achieve this, the artist quickly sets up a coarse simulation and uses the output geometry to guide the main features of a full resolution simulation.
Several approaches modify a high-resolution smoke simulation using optimization~\cite{Nielsen2009,Nielsen2010}, patterns extracted as skeleton~\cite{Yuan2011}, or sparse sampling~\cite{Huang2013}.
For liquid simulations, Nielsen and Bridson~\cite{Nielsen2011} propose to restrict the high resolution simulation to a thin layer around a guiding coarse animation.

Although each of these approaches are able to successfully guide a fluid simulation, they do not enable direct control of the resulting fluid. Designing precise timing or feature scaling would therefore still require iterative trial-and-error steps to converge toward a desired animation.
Few attempts have been made to enable direct control on the simulation. Schpok et al.~\cite{Schpok2005} proposed to extract and parameterize features such as vortices, uniform advection, sinks, and sources to allow the user to modify the parameters in a smoke simulation. In the context of liquid simulation, Pan et al.~\cite{Pan2013} propose a method to deform wave shapes by sketching their profiles. This approach enables direct spatial deformation but does not allow temporal editing, and the simulation needs to be re-computed from the modified frame onwards.

Editing fluid animation is usually handled through indirect control, by adjusting physically based simulation parameters and degrees of freedom~\cite{stam1999,ihmsen2014}, or by editing control parameters of procedural generation models for ocean and waves~\cite{Fournier1986,hinsinger2002,Tessendorf2004,jeschke2015water}. 

Direct editing of simulation has been addressed in the contexts of rigid bodies and of deformable objects~\cite{Chenney2000,wojtan2006keyframe,Twigg2007,Barbic2009,Barbic2012,Schulz2014,Li2014}, but only few works have focused on fluid surfaces, due to the fact that the constantly changing shape and topology makes the output geometry inaccessible to standard deformation tools.

In this section, we focus on the design of fluid animation: firstly by \emph{controlling the simulation}, secondly by \emph{editing the animation}. 

\subsubsection{Open problems}