\chapter[Conclusion]{Conclusion}
\label{chap:conclusion}

\Lettrine{I}{n} this manuscript, we detailed several approaches to simulate and control mechanical models in computer graphics. 
Here, we summarize the different contributions, their limitations and perspectives of future work.

\section{Summary of the contributions}
We can divide this work into two main types of contributions.
\paragraph*{}
The first focus was the study of new techniques towards the efficient simulation of mechanical models. 
Aside from our state of the art on adaptive models, we proposed two approaches tackling this goal: 
First we introduced a new adaptive method that allows to speed-up simulations with very few modifications of an existing simulator. 
In contrast with the usual re-sampling schemes, the method only requires to adapt the time integrator. 
We demonstrated the efficiency of this method on particle-based fluid and on cloth simulation.
Second we proposed a method to handle detailed topological changes while keeping a very low number of degrees of freedom. 
This was made possible by a dynamic update of the shape functions associated to each degrees of freedom in order to take into account the new topology into the dynamics. 
Combined with an incremental update of the simulation data and an interpolation of the positions and velocities of the degrees of freedom, this method allows to simulate detailed cutting of thin sheets at interactive rates.

Our second focus was on the control of physics-based animation. We proposed a new system to edit liquid animation. 
Instead of enforcing a simulation with user constraints, we proposed to build tools that would allow the user to edit existing liquid animation results. 
Starting from a simple sequence of meshes representing the surface of the liquid, the user can select coherent chunks of the animation and edit them with standard paradigms inspired from surface modeling such as copy, edit, paste and temporal tools from movie editing such as temporal remapping. 
Pushing forward this space-time deformation framework, the user can re-use parts of an animation in other animated scenes.

\section{Limitations and future work}

\paragraph{Adaptively Restrained Particles} First, the method could be easily improved by exploring more visual criteria to adapt the simulation. 
In particular, the use of criteria varying in space and time was not explored and the robustness of the method to such variations is not clear. 
Second, the interest of the method is restrained to cases where parts of the simulation do not move. 
Closely related to this freezing technique, we would like to investigate the simplification of parts of a model that behave rigidly. 
In many cases, elastic deformations are concentrated near the surface of the object while the interior may have a rigid like behavior. 
We think it would be interesting to identify and simplify these parts.

\paragraph{Detailed cutting of thin sheets} There are a large number of directions that could be investigated to improve this method. 
First, the popping artifacts that can occur when the number of degrees of freedom should be prevented. 
Pfaff et al.~\cite{Pfaff2014} proposed an optimization-based approach to reduce popping as much as possible in fracture scenario.
We could start from their work to see if it would fit the frame-based model.
Second, fracture could be integrated by combining a sparse computation of the stress tensor and a procedural method to generate details along the crack. 
Third, the extension of our method to volumetric objects would require a robust method to build the non-manifold grid we use to update the shape functions. 
In related works the building of such a grid is made using the volumetric discretization of the object~\cite{Mitchell2015a,Mitchell2015b}. 
In our case, it is essential to build the grid only from the surface of the object. 
Finally, we would like to incorporate plastic deformations while keeping a very low number of degrees of freedom and bring our model to real time performance.

\paragraph{Fluid sculpting} The actual system cannot handle large simulations involving very turbulent flows. 
The first reason is the memory size of such animations. 
In order to alleviate this problem, we could interact with a low resolution version of the animation and then apply all the transformations of the user to the high resolution version. 
The second reason is that waves resulting from turbulent flows may present complex topology which could not be faithfully captured by our approach based on displacement field.
We think that state of the art method in surface modeling are not sufficient to deal with such surfaces. 
Instead we would like to use the mesh of these waves as a target to the surface where the waves are being pasted and use a constant volume deformation tool. 
This would help solving another limitation of our work which is the physical consistency of the deformation. 
When pasting waves or droplets, there are no guarantee except the user's experience that the result would look realistic. 
With constant volume deformation we hope that physical consistency would be enforced. 
Finally we think that we only scratched the surface of the all the temporal edit operations. 
Building temporal tools that would enforce consistency from one frame to another is an exciting avenue for future research.
