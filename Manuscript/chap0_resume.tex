\chapter*[R\'esum\'e]{R\'esum\'e}

\section*{\Large{Simulation et contrôle de phénomènes physiques}}

En informatique graphique les phénomènes physiques simulés pour la création d'animations, de jeux vidéos ou la conception d'objets sont de plus en plus complexes :
tout d'abord en terme de coût de calcul, l'échelle des simulations étant de plus en plus importante ;
ensuite en terme de complexité des phénomènes eux-mêmes qui requièrent des modèles permettant de changer d'état et de forme. 
Cette complexité grandissante introduit de nouveaux défis quand il s'agit d'offrir à un utilisateur un contrôle sur ces simulations à grande échelle. 
Dans de nombreux cas, ce contrôle est réduit à un cycle d'essais et d'erreurs pour déterminer les paramètres de la simulation qui satisferont au mieux les objectifs de l'utilisateur.

Dans cette thèse, nous proposons trois techniques pour répondre en partie à ces défis. 
Tout d'abord nous introduisons un nouveau modèle adaptatif permettant de réduire le temps de calcul dans des simulations Lagrangiennes de particules. 
À l'inverse des méthodes de ré-échantillonnage, le nombre de degrés de liberté reste constant au cours de la simulation. 
La méthode est ainsi plus simple à intégrer dans un simulateur existant et la charge mémoire est constante, ce qui peut être un avantage dans un contexte interactif. 
Ensuite, nous proposons un algorithme permettant de réaliser la découpe détaillée d'objets fins et déformables. 
Notre méthode s'appuie sur une mise à jour dynamique des fonctions de forme associées à chaque degré de liberté, permettant ainsi de conserver un nombre de degrés de liberté très faible tout en réalisant des changements topologiques détaillés. 
Enfin, nous nous intéressons au contrôle d'animations de liquide en s'inspirant des méthodes d'édition interactive de formes en modélisation 3D. 
Dans ce système, l'utilisateur travaille directement avec le résultat d'une simulation, c'est-à-dire une suite de maillages représentant la surface du liquide. 
Des outils de sélection et d'édition spatio-temporelle inspirés des logiciels de sculpture de formes statiques lui sont proposés.
