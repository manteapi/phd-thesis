\chapter*[Remerciements]{Remerciements}

Je remercie tout d'abord François Faure et Marie-Paule Cani pour m'avoir donné la chance de réaliser ce doctorat, 
pour m'avoir encouragé tout au long de ces quatre ans et avoir partagé avec autant d'enthousiasme leur passion pour la recherche.
Je remercie François pour m'avoir tant appris sur la simulation physique et pour m'avoir aidé à détricoter méticuleusement tant de questions scientifiques.
Je remercie Marie-Paule pour sa joie de vivre à chacune de nos réunions, ses formidables intuitions et la générosité avec laquelle elle transmet ses connaissances, ses idées et ses conseils.

J'ai eu la chance de collaborer avec de nombreuses personnes d'horizons très différents pendant les projets de cette thèse.
Merci à Stéphane Redon pour sa patience et son implication sur mon premier projet de recherche et sur la rédaction au long cours de l'état de l'art sur les méthodes adaptatives. Merci à Weilun Sun et James O'Brien pour ces trois superbes mois passés à Berkeley, c'est certainement l'expérience de recherche la plus forte que j'aurais connue pendant cette thèse. Merci à Paul Kry pour son dynamisme, son retour si pertinent sur mes travaux et son aide inestimable sur mes fautes d'anglais. Merci aussi à Rahul Narain et Chris Wojtan pour les nombreuses discussions autour des méthodes adaptatives. Je tiens également à remercier Chris pour m'avoir accueilli à l'IST pendant trois semaines, d'avoir été aussi ouvert et attentif sur toutes les pistes que l'on envisageait pour le contrôle de liquide. J'ai été très impressionné par le dynamisme qu'il insuffle à son équipe de recherche, ce fut un plaisir de partager leur quotidien. Merci à Ulysse Vimont, pour son petit grain de folie et toutes ses heures passées à faire avancer notre projet de sculpture de liquides. Je n'aurais pas pu le réaliser sans toi et le passage à l'IST aurait été très différent. Enfin, j'ai été très heureux que l'on puisse confronter autant d'idées toujours dans un esprit de collaboration. 

La thèse m'a permis de découvrir le plaisir d'enseigner. Je souhaiterais remercier tous les étudiants que j'ai rencontrés au cours de ces années.
Ils m'ont permis de trouver un équilibre quand la thèse n’avançait pas, de découvrir d'autres pistes de recherche et de garder le lien avec ces années que j'ai passées à l'ENSIMAG. Je tiens tout spécialement à remercier les groupes de projet de spécialité, c'est toujours une expérience forte de suivre un petit groupe d'étudiants pendant quatre semaines. Merci en particulier à Thibault Lejemble, Amélie Fondevilla, Thibault Blanc-Beyne et Nicolas Durin, qui ont su pousser leur projet jusqu'à obtenir une publication dans une conférence internationale. Merci également à Mickaël Ly, Ilyes Kacher, Mathieu Stoffel et Maxence Hammen pour leur motivation sans failles, leur bonne humeur et pour avoir su garder le contact une fois le projet terminé. Merci à Camille Schreck pour sa combinaison très personnelle de l'humour et du sérieux et son aide dans l'encadrement d'un projet de spécialité. Merci à Thomas Delame pour son implication et sa détermination à donner un nouveau souffle aux TP du cours de Graphique 3D.


Merci à tous les membres de IMAGINE et MAVERICK avec qui j'ai partagé tant de bons moments à discuter, troller, jouer au baby-foot ou au ping-pong.
Merci à Léo pour sa présence indéfectible que ce soit pour des questions de mathématiques, des relectures d'articles, du soutien moral ou des bières : je n'imagine même pas comment ce serait passé ces quatre ans sans toi.
Merci à Hugo pour sa folie et son incroyable capacité à rendre réel ce que l'on n'ose imaginer. Je pense que je me souviendrais encore très longtemps de notre rapide passage aux studios Disney de Los Angeles et à ces quelques minutes passées à regarder des rushs avec une des légendes de l'animation. Rien de tout cela ne serait arrivé sans toi.
Merci à Quentin et Ali pour tous ses bons moments passés pendant notre viré en Californie après SIGGRAPH.
Merci à Damien pour sa sagesse dont j'ai tant aimé me moquer mais sans laquelle je me serais senti bien démuni.
Merci à Antoine pour sa bonne humeur et son écoute, plus d'une fois ça m'a permis de décompresser et de repartir à l'attaque.
Merci à Matthieu, Romain et Thomas pour toutes les conversations Sofaïenne.

Merci aux membres du jury pour leur relecture du manuscrit et leur implication tout au long de la soutenance.

Je remercie mes amis et ma famille pour m'avoir supporté et encouragé pendant ces quatre ans.
Mon épouse, Nassima, pour avoir été à mes côtés à chaque instant, qu'il soit bon ou mauvais.
Mes parents, Marie-Hélène et Gabriel, et mes sœurs, Anne-Elisabeth et Claire-Lucie, pour leur soutien et toutes ses bouffées d'air qui m'ont aidé à poursuivre ce doctorat.
